\section*{Aufgabe 1}

%Die Simulationen wurden folgende Einstellungen verwendet, sofern nicht anders
%gekennzeichnet:
%\begin{itemize}
%    \item $N = 16$
%    \item $r_\text{c} = \frac{L}{2}$
%    \item $h = \num{0,01}$
%    \item $k_\text{B} = 1$
%    \item $m_\text{i} = 1 \quad \forall i = 1, ..., N$
%    \item Anzahl Freiheitsgrade $N_\text{f} = 2 N - 2$
%\end{itemize}
%Längen werden zudem in Einheiten von $\sigma$ und Zeiten in Einheiten von $\tau$ gemessen.
%Energien und $k_\text{B} T$ werden in Einheiten von $\varepsilon$ gemessen.

\subsection*{a) Initialisierung}

Zu Beginn werden die Orte und Geschwindigkeiten der Teilchen in einer $[0,L] \cdot [0,L]$ Box initialisiert.
Dabei werden zufällige Startgeschwindigkeiten im Bereich \([0, 10]\) gewählt. 
Zusätzlich wird die Schwerpunktsbewegung der Teilchen auf Null gesetzt.
Um die Geschwindigkeiten entsprechend der Temperatur umskalieren zu können, wird folgende Rechnung betrachtet:
\begin{align*}
    T_0 &=
    \frac{2}{k_\text{B} N_\text{f}} \sum_{i=1}^N \frac{\vec{p}_\text{i}^2}{2 m_\text{i}} \\
    &= \frac{1}{N_\text{f}} \sum_{i=1}^N \left(v_\text{x}^2 + v_\text{y}^2\right) \\
    &= \frac{1}{N_\text{f}} \sum_{i=1}^N
        \left(a^2 \tilde{v}_\text{x}^2 + a^2 \tilde{v}_\text{y}^2\right) \\
    &= a^2 \frac{1}{N_\text{f}} \sum_{i=1}^N
        \left(\tilde{v}_\text{x}^2 + \tilde{v}_\text{y}^2\right) \\
    &= a^2 \tilde{T}.
\end{align*}
Aus ihr folgt, dass die Geschwindigkeiten mit 
\begin{equation*}
    \frac{1}{a} = \sqrt{\!\left(\frac{\tilde{T}}{T}\right)}
\end{equation*}
skaliert werden müssen.

\FloatBarrier
\subsection*{b) Äquilibrierung}

Die Dynamik des Systems wird nun mithilfe des Geschwindigkeits-Verlet-Algorithmus ermittelt.
Die dabei benötigte Beschleunigung der einzelnen Teilchen ergibt sich aufgrund normierter Massen von 1
direkt aus der Kraft, welches sich wiederum aus dem Lennard-Jones Potential
\begin{align*}
    \vec{F}_\text{i}
    &= \sum_{i \neq j} \vec{F}_\text{ij}
    = - \sum_{i \neq j} \sum_{\vec{n} \in \mathbb{Z}^2}
        \frac{\vec{r}_\text{ij} + L \vec{n}}{\left|\vec{r}_\text{ij} + L \vec{n}\right|}
        V'\left(\left|\vec{r}_\text{ij} + L \vec{n}\right|\right) \\
    \text{Cutoff } r_\text{c}
    &= \frac{1}{2} L < \left|\vec{r}_\text{ij} + L \vec{n}\right| \\
    \vec{F}\!\left(\vec{r}\right)
    &= - \vec{\nabla} V\left(r\right)
    = - 24 \frac{\vec{r}}{r^2}
        \left[2 \left(\frac{1}{r}\right)^{-12} - \left(\frac{1}{r}\right)^{-6}\right]
\end{align*}
ergibt.
Dabei stellt der Cutoff $r_\text{c}$ sicher, dass ein Teilchen $i$ nur einmal mit einem (Bild)Teilchen $j$
wechselwirkt.

Die während der Äquilibrierung ergeben sich die Schwerpunktsgeschwindigkeit $\vec{v}_\text{S}$,
die Temperatur $T$, die potentielle und die kinetische Energie
($E_\text{pot}$, $E_\text{kin}$) mithilfe von
\begin{align*}
    \vec{v}_\text{S} &= \frac{1}{N} \sum_{i=1}^N \vec{v}_\text{i} \\
    E_\text{pot} &= \sum_{i=1}^N \sum_{j=i+1}^N
        V\!\left(\left|\vec{r}_\text{i} - \vec{r}_\text{j}\right|\right) \\
    E_\text{kin} &= \sum_{i=1}^N \frac{1}{2} \vec{v}_\text{i} \\
    T &= \frac{2}{N_\text{f}} E_\text{kin}
\end{align*}
berechnet.

Diese genannten größen werden in Abbildung \ref{fig:T1}, \ref{fig:v1} und \ref{fig:e1}
dargestellt. Da sich die Temperatur aus der kinetischen Energie berechnet, zeigen sie
denselben Verlauf.

\begin{figure}
    \centering
    \includegraphics[width=0.45\textwidth]{A1/build/aequi1_E.pdf}
    \includegraphics[width=0.45\textwidth]{A1/build/aequi1_EE.pdf}
    \caption{ $T_0 = 1\:\varepsilon$.}
    \label{fig:e1}
\end{figure}

\begin{figure}
    \centering
    \includegraphics[width=0.45\textwidth]{A1/build/aequi1_T.pdf}
    \includegraphics[width=0.45\textwidth]{A1/build/aequi1_TT.pdf}
    \caption{$T_\text{0} = 1\:\varepsilon$.}
    \label{fig:T1}
\end{figure}

\num{1e-15}.
\begin{figure}
    \centering
    \includegraphics[width=0.45\textwidth]{A1/build/aequi1_V.pdf}
    \includegraphics[width=0.45\textwidth]{A1/build/aequi1_VV.pdf}
    \caption{$T_\text{0} = 1\:\varepsilon$.}
    \label{fig:v1}
\end{figure} 

\FloatBarrier
% \subsection*{c) Messung}
%
% Nach der Äquilibrierungsphase wird die Dynamik des Systems für \num{5e4} Zeitschritte
% berechnet. Diese Berechnung funktioniert analog zur Äquilibrierung.
% In jedem Schritt wird wieder die Temperatur bestimmt und abgespeichert, das
% python skript ermittelt dann den Mittelwert der Temperatur.
% Aus Zeitgründen wurde auf eine Berechnung der Paarkorrelation verzichtet.
%
% In den Abbildungen \ref{fig:messung_T=1_temp} bis \ref{fig:messung_T=1e-2_temp}
% sind die zeitlichen Verläufe der Temperatur
% bei verschiedenen Starttemperaturen dargestellt.
% Hierbei ist zu beachten, dass bei $T = 100\:\frac{\varepsilon}{k_\text{B}}$
% die Startgeschwindigkeiten so groß sind, dass die Schrittweite von \num{0.01}
% auf \num{0.001} verringert wurde.
% \begin{figure}
%     \centering
%     \includegraphics[height=8cm]{build/messung_T1_temp.jpg}
%     \caption{Temperaturverlauf der Messung bei $T_\text{init} = 1\:\varepsilon$.}
%     \label{fig:messung_T=1_temp}
% \end{figure}
% \begin{figure}
%     \centering
%     \includegraphics[height=8cm]{build/messung_T1e2_temp.jpg}
%     \caption{Temperaturverlauf der Messung bei $T_\text{init} = 100\:\varepsilon$.}
%     \label{fig:messung_T=1e2_temp}
% \end{figure}
% \begin{figure}
%     \centering
%     \includegraphics[height=8cm]{build/messung_T1e-2_temp.jpg}
%     \caption{Temperaturverlauf der Messung bei $T_\text{init} = 0.01\:\varepsilon$.}
%     \label{fig:messung_T=1e-2_temp}
% \end{figure}
%
% Die Mittelwerte der Temperaturen ergaben sich zu
% \begin{align*}
%     T_\text{init} = 10^0: \quad T_\text{mean} &= \num{1.48(13)} \\
%     T_\text{init} = 10^2: \quad T_\text{mean} &= \num{96(5)} \\
%     T_\text{init} = 10^{-2}: \quad T_\text{mean} &= \num{0.7(1)} \\
% \end{align*}
% Die Temperaturen weichen nur wenig von der Starttemperatur ab.
% Dies ist auch sinnvoll, da keine weitere Energie in das System hinein oder aus dem
% System heraus geflossen ist. Aufgrund der surrealen Startkonfiguration verändert sich
% jedoch die potentielle Energie und damit ist aufgrund der Energieerhaltung auch eine
% kleine Abweichung der kinetischen Energie und damit Temperatur von dem initialen Wert
% zu erwarten.
%
% \FloatBarrier
% \subsection*{d) Thermostat}
%
% Der Übergabewert \texttt{thermostat} an die Funktion \texttt{md\_simulation} legt fest,
% ob ein isokinetischer Thermostat in das System eingebaut werden soll.
% Ist dies der Fall, werden nach jedem Schritt die Geschwindigkeiten so skaliert, dass
% das System wieder auf der initialen Temperatur gehalten wird.
% In Abbildung \ref{fig:thermostat_equi_energy} ist der Verlauf der Energien während der
% Äquilibrierungsphase für $T_\text{init} = 0.01$ dargestellt.
% Durch die Anpassung der Geschwindigkeiten gilt keine Energieerhaltung mehr
% und die potentielle Energie wird im Laufe der Zeit minimiert.
%
% Da die Temperatur recht klein gewählt ist, bewegen sich die Teilchen nicht viel
% und sollten eine periodische Anordnung einnehmen.
% In Abbildung \ref{fig:endkonfig} ist die Konfiguration nach der Messung dargestellt.
% Es zeigt sich, dass kleinere Muster erkennbar sind.
% Dies spricht für eine feste Phase der Teilchen.
% \begin{figure}
%     \centering
%     \includegraphics[height=8cm]{build/thermostat_equi_energy.jpg}
%     \caption{Energieverlauf während der Äquilibrierungsphase bei isokinetischem Thermostat.}
%     \label{fig:thermostat_equi_energy}
% \end{figure}
% \begin{figure}
%     \centering
%     \includegraphics[height=6cm]{build/endkonfig.jpg}
%     \caption{Endkonfiguration mit 16 Teilchen bei isokinetischem Thermostat.}
%     \label{fig:endkonfig}
% \end{figure}
