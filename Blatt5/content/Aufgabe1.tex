\section*{Aufgabe 1}
\subsection*{1)} 
Eine direkte Rechnung über die Funktion 

\begin{equation}
    F_j = \sum_{l=0}^{N-1}\Omega_N^{j, l} f_l
\end{equation}
mit 
\begin{align}
    \Omega_N^{j, l} &= (e^{2 \pi i \cdot \frac{j}{N}})^{l} = \left((-1)^{\frac{2j}{N}}\right)^{l} \\
    f_l &= \sqrt{1+l} \\
    N &= 2^m
\end{align}
Für $m = 3$ ergibt sich der Vektor
\begin{equation}
    f = 
    \begin{pmatrix} 
        -&1.3823 -& i \cdot 2.23154 \\
        -&1.1417 -& i \cdot 0.964724 \\
        -&1.0898 -& i \cdot 0.404137 \\
        -&1.0782 +& i \cdot 1.36888e-15 \\
        -&1.0898 +& i \cdot 0.404137 \\
        -&1.1417 +& i \cdot 0.964724 \\
        -&1.3823 +& i \cdot 2.23154 \\
         &16.306 -& i \cdot 1.65924e-14
    \end{pmatrix}
\end{equation}
Für $m = 4$ ergibt die direkte Rechnung:
\begin{equation}
    f =
    \begin{pmatrix}
        -&2.85108 -& i \cdot 7.02149  \\
        -&2.01863 -& i \cdot 3.57652 \\
        -&1.80909 -& i \cdot 2.26138 \\
        -&1.72579 -& i \cdot 1.52489 \\
        -&1.68555 -& i \cdot 1.02385 \\
        -&1.66461 -& i \cdot 0.636399 \\
        -&1.65427 -& i \cdot 0.306027 \\
        -&1.65114 +& i \cdot 1.71106e-14 \\
        -&1.65427 +& i \cdot 0.306027 \\
        -&1.66461 +& i \cdot 0.636399 \\
        -&1.68555 +& i \cdot 1.02385 \\
        -&1.72579 +& i \cdot 1.52489 \\
        -&1.80909 +& i \cdot 2.26138 \\
        -&2.01863 +& i \cdot 3.57652 \\
        -&2.85108 +& i \cdot 7.02149 \\
         &44.4692 -& i \cdot 1.23835e-13
    \end{pmatrix}
\end{equation}
Die schnelle Fouriertransformation liefert folgende Ergebnisse:
$m = 3$:
\begin{equation}
    f =
    \begin{pmatrix}
        foo  \\
        bar 
    \end{pmatrix}
\end{equation}

$m = 4$:
\begin{equation}
    f =
    \begin{pmatrix}
        foo  \\
        bar 
    \end{pmatrix}
\end{equation}

\subsection*{2)} 
\paragraph*{a)}
arrogance and total loss of all senses!
\paragraph*{b)}
Für die analytische Lösung der Fouriertransformation von 
\begin{equation}
    f(x) = \exp{\left( - \frac{x^2}{2} \right)}
\end{equation}
Ergibt sich 
\begin{equation}
    F(k) = \frac{1}{2\pi} \int_{-\infty}^{\infty} \exp{\left( - \frac{x^2}{2} \right)} \exp{\left(i k x \right) } dx = \frac{1}{\sqrt{2 \pi}} \exp{\left( - \frac{k^2}{2} \right)}
\end{equation}
Der Vergleich zwischen der analytischen und numerischen Lösung für ein Intervall von $\left[-10, 10\right]$ befindet sich in Abbildung \ref{fig:F}.
%\begin{figure}
%    \centering
%    \includegraphics[]{}
%    \caption{Numerische und analytische Lösung}
%    \label{fig:F}
%\end{figure}

\subsection*{3)}
In dieser Teilaufgabe werden die komplexen Fourierkoeffizienten $c_n$ von 
\begin{equation}
    f(x) = \begin{cases}
        -&1, x \in \left[ - \pi, 0\right] \\
         &1, x \in \left(0 , \pi\right)
    \end{cases}
    \label{eqn:recht}
\end{equation}
für $ m = 7 $ bestimmt. 
Analytisch ergibt sich $c_n$ durch 
% Quelle: http://www.chemgapedia.de/vsengine/vlu/vsc/de/ma/1/mc/ma_12/ma_12_02/ma_12_02_04.vlu/Page/vsc/de/ma/1/mc/ma_12/ma_12_02/ma_12_02_11.vscml.html
\begin{align}
    c_n = \frac{1}{2} \left( a_n + i b_n \right) \\
    a_n = \frac{2}{L} \int_L f(x) \cdot \cos(  2\pi \frac{n}{L}) dx \\
    b_n = \frac{2}{L} \int_L f(x) \cdot \sin(  2\pi \frac{n}{L}) dx
\end{align}
Mit Gleichung \eqref{eqn:recht} ergibt sich 
\begin{align}
    a_n = 0 \text{  für alle n} \\
    b_n = \frac{1}{n} \cdot \frac{4}{\pi} \text{    für ungerade n} \\
    c_n = i \cdot \frac{1}{n} \cdot \frac{4}{\pi} \text{    für ungerade n}
\end{align}
