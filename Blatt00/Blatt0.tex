\documentclass[12pt,a4paper]{article}
\usepackage[utf8]{inputenc}
\usepackage[german]{babel}
\usepackage[T1]{fontenc}
\usepackage{amsmath}
\usepackage{amsfonts}
\usepackage{amssymb}
\usepackage{graphicx}
\begin{document}
\section{Blatt 0}
\subsection*{Aufgabe 0}
\subsubsection*{a)}
Die Stabilität eines numerischen Verfahrens oder Problems beschreibt seine Empfindlichkeit gegenüber kleinen Fehler in den Anfangsbedingung. Werden diese mit zunehmenden Iterationsschritten stark vergrößert, spricht man von numerischer Instabilität.
\subsubsection*{b)}
Eine höhere Genauigkeit, z.B. die Berücksichtigung höherer Ordnungen kann zu 
\subsection*{Aufgabe 1}
Das Distributivgesetz kann aufgrund der begrenzten Genauigkeit von Gleitkommazahlen entstehen.\\
Bsp. mit $t=2$ und $l=1$
\begin{align*}
a &= 0,50\cdot 10^9,\\
b &= 0,60\cdot 10^9,\\
c &= 0,10\cdot 10^{-3}
\end{align*}
\begin{align*}
(a+b)\cdot c &= \underbrace{(0,50\cdot 10^9 +0,60\cdot 10^9)}_{=0.11\cdot 10^{10},exponent overflow\rightarrow 0,99\cdot 10^{9}}\cdot 0,10\cdot 10^{-3} = 0,99\cdot 10^5}\\
a\cdot c + b\cdot c &= 0,50\cdot 10^5 + 0,60\cdot 10^5 = 0,11\cdot 10^6
\end{align*}
\subsection*{2)}
\subsubsection*{a)}
Das Problem kann umgeschrieben werden zu:
\begin{align*}
\frac{1}{\sqrt{x}} - \frac{1}{\sqrt{x+1}} &= \frac{\sqrt{x+1}-\sqrt{x}}{\sqrt{x^2+x}}\\
&= 
\end{align*}
\end{document}


%♥