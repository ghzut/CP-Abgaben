\subsection*{Aufgabe 2}

\subsubsection*{a)}

Es soll das elektrostatsische Potential auf der x-Achse gemäß
\begin{equation}
  \Phi(x) = \frac{1}{4 \pi \varepsilon_0} \int \int \int \frac{\rho(x', y', z')}{((x-x')^2+y'^2+z'^2)^{\sfrac{1}{2}}} \mathrm{d}x'\mathrm{d}y'\mathrm{d}z'
  \label{eqn:phi}
\end{equation}
mit einer homogenen Ladungsverteilung
\begin{equation*}
  \rho(x, y, z) = \begin{cases}
  \rho_0 & |x|,|y|,|z| < a \\
  0 &\, \text{sonst}
\end{cases}
\end{equation*}
berechnet werden.

Das gegebene Integral wird dafür zunächst einheitenlos gemacht. Dafür werden folgende Ersetzungen vorgenommen:
\begin{align*}
    \tilde{x} &= \frac{x}{a} \\
    \tilde{\Phi} &= \frac{4 \pi \varepsilon_0}{\rho_0 a^2} \Phi.
\end{align*}
Zusätzlich wird \(a = 1\) gesetzt. Damit ergibt sich
\begin{equation*}
  \tilde{\Phi}(x) = \int_{-1}^1 \int_{-1}^1 \int_{-1}^1 \frac{1}{((x-x')^2+y'^2+z'^2)^{\sfrac{1}{2}}} \mathrm{d}x'\mathrm{d}y'\mathrm{d}z'.
\end{equation*}

Das Potential außerhalb des Würfels kann mithilfe einer Multipolentwicklung zur ersten nicht verschwindenden Ordnung, also dem Monopol-Moment, genähert werden
\begin{equation*}
  \tilde{\Phi}(x) \: \propto \: \frac{1}{\rho_0 |x|} \int \int \int \rho(x', y', z') \mathrm{d}x'\mathrm{d}y'\mathrm{d}z' = \frac{1}{|x|}.
\end{equation*}
In Abbildung \ref{fig:aus_a} sind \(\tilde{\Phi}(x)\) sowie die Näherung aufgetragen. Daran ist zu sehen, dass die Näherung den Trend der Werte sehr gut abbildet.
In Abbildung \ref{fig:inn_a} ist das Potential innerhalb des Würfels aufgetragen.
\begin{figure}
  \centering
  \begin{subfigure}[b]{0.45\textwidth}
      \includegraphics[width=\textwidth]{A2/build/innerhalb_a.pdf}
      \caption{Potential innerhalb des Würfels.}
      \label{fig:inn_a}
    \end{subfigure}
    ~ %add desired spacing between images, e. g. ~, \quad, \qquad, \hfill etc.
    %(or a blank line to force the subfigure onto a new line)
    \begin{subfigure}[b]{0.45\textwidth}
      \includegraphics[width=\textwidth]{A2/build/ausserhalb_a.pdf}
      \caption{Potential außerhalb des Würfels.}
      \label{fig:aus_a}
    \end{subfigure}
    \caption{Berechnung von \(\tilde{\Phi}(x)\).}\label{fig:a}
\end{figure}

\subsubsection*{b)}

Nun soll das elektrostatsische Potential \eqref{eqn:phi} mit der Ladungsverteilung
\begin{equation*}
  \rho(x, y, z) = \begin{cases}
  \rho_0 \frac{x}{a} & |x|,|y|,|z| < a \\
  0 &\, \text{sonst}
  \end{cases}
\end{equation*}
bestimmt werden. Da \(x\) und \(a\) die gleiche Einheit besitzen, können dieselben Ersetzungen wie in a) verwendet werden, um das Integral einheitenlos zu machen. Auch hier wird \(a=1\) gesetzt.
Damit ergibt sich
\begin{equation*}
  \tilde{\Phi}(x) = \int_{-1}^1 \int_{-1}^1 \int_{-1}^1 \frac{x}{((x-x')^2+y'^2+z'^2)^{\sfrac{1}{2}}} \mathrm{d}x'\mathrm{d}y'\mathrm{d}z'.
\end{equation*}
Die erste nicht verschwindende Ordnung der Multipolentwicklung stellt dieses Mal das Dipol-Moment dar. Das Potential außerhalb des Würfels kann damit als
\begin{equation*}
  \tilde{\Phi}(x) \: \propto \: \frac{\vec{e}_\text{r}}{\rho_0 x^2} \int \int \int \vec{r}' \rho(x', y', z') \mathrm{d}x'\mathrm{d}y'\mathrm{d}z' = \frac{1}{x^2}
\end{equation*}
genähert werden.
In Abbildung \ref{fig:aus_b} sind \(\tilde{\Phi}(x)\) und die Näherung aufgetragen. Dieses Mal beschreibt die Näherung die tatsächlichen Werte nur sehr schlecht.
\begin{figure}
  \centering
  \begin{subfigure}[b]{0.45\textwidth}
      \includegraphics[width=\textwidth]{A2/build/innerhalb_b.pdf}
      \caption{Potential innerhalb des Würfels.}
      \label{fig:inn_b}
    \end{subfigure}
    ~ %add desired spacing between images, e. g. ~, \quad, \qquad, \hfill etc.
    %(or a blank line to force the subfigure onto a new line)
    \begin{subfigure}[b]{0.45\textwidth}
      \includegraphics[width=\textwidth]{A2/build/ausserhalb_b.pdf}
      \caption{Potential außerhalb des Würfels.}
      \label{fig:aus_b}
    \end{subfigure}
    \caption{Berechnung von \(\tilde{\Phi}(x)\).}\label{fig:b}
\end{figure}
